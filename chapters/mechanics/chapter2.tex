
\chapter{Motion along a Straight Line}

\section{Discussion Questions}

% 2.1
\discussion{Speed.  Velocity has a direction as well as a magnitude.}

% 2.2
\discussion{Since the dots appear to be increasing by what appears a multiple of the previous distance I would say it is graph (d).}

% 2.3
\discussion{Yes, if the velocity and acceleration have ``different signs''.  This is because the velocity is ``slowing down'' and will eventually reach zero, then start ``increasing again''.  You can also look at the formula for velocity with constant acceleration to see why.  It cannot reverse twice though as a straight line cannot pass through the x-axis twice.}

% 2.4
\discussion{Either when the velocity is constant or special cases when the acceleration changes direction and after a period of time the average velocity will match the instantaneous velocity.}

% 2.5
\discussion{ (b) Yes.  When the velocity and acceleration are in ``opposite directions''.  Yes, if they both point in the ``same direction''.}

% 2.6
\discussion{Constant acceleration.  Anytime you start from 0 speed and accelerate then end up at 0 speed will the speed equal the magnitude of the velocity at some point.}

% 2.7
\discussion{Average velocities are the same but in opposite directions.}

% 2.8
\discussion{You could argue that you would be able to ticket all parked cars that people who speed pass by.}

% 2.9
% This question is no very clear.  I assume they mean instantaneous velocity when they say velocity.
\discussion{No, because displacement is 0, $\Delta{x}$ = zero, then by definition the average velocity is 0.  For the second yes;  $x =sin(t)$, then $y=cos(t)$.  When t = 0 displacement is 0 but velocity is 1.}

% 2.10
\discussion{Yes. If you assume the object is already moving and there is no acceleration then the object will keep moving.}

% 2.11
\discussion{180km/hr * 1h/60m * 1m = 3km. 6.0 seconds is 1/10th of a minute, so 1/10th of 3km, 0.3km}

% 2.12
\discussion{360km/hr, since this is linear half the time means twice the speed.}

% 2.13
\discussion{}

% 2.14
\discussion{}

% 2.15
\discussion{}

% 2.16
\discussion{}

% 2.17
\discussion{}

% 2.18
\discussion{}

% 2.19
\discussion{}

% 2.20
\discussion{}

% 2.21
\discussion{}

% 2.22
\discussion{}

\section{Exercises}

\section{Problems}

\section{Challenge Problems}