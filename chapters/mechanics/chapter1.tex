
\chapter{Units, Physical Quantities, and Vectors}

\section*{Discussion Questions}

% 1.1
\discussion{We just need a single experiment to disprove a theory.  However, there is no number of experiments to prove a theory, just increase confidence.  This breaks down with a finite number of options.}

% 1.2
\discussion{That is not possible as the tangent function takes in degrees and not distances.}

% 1.3
\discussion{My height in imperial is 6' 2", which is 187.96 cm.  My weight is 178 lbs in imperial and therefore it is 792.1 N}

% 1.4
\discussion{Yes.  Currently the mass is derived directly from these.  Other quantities are based off of properties events in nature, but the mass is derived by humans.}

% 1.5
\discussion{You could use pulsars as a way to measure time as they have a very precise rotation.  Less precise measures could be position of sun in the sky to determine the hour of day.}

% 1.6
\discussion{Let's assume an 8 by 11 inch sheet of paper.  Then you can cut it into squres of 1x1 inch and then stack them up.  Measure that value and then divide it by 88 to approximate the thickness of a piece of paper.}

% 1.7
\discussion{An axiom is a self consistent statement.  It does not mean that it matches reality.  In terms of physics you could argue that an axiom is only true if the assumptions you are making are true.  A theory is only true as far as the experiments have shown it to be the best match based on observations.}

% 1.8
\discussion{The units of volume are kilogram per meter-cubed and gram per centimeter-cubed.}

% 1.9
\discussion{Joe is neither accurate not precise, Moe is accurate but not precise, Flo is both accurate and precise.}

% 1.10
\discussion{}

\section*{Exercises}

\exercise{}