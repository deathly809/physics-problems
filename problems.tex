\documentclass[oneside]{book}

\usepackage{enumerate}% http://ctan.org/pkg/enumerate

% Pictures
\usepackage{graphicx}
\usepackage{grffile}
\usepackage{mwe}
\usepackage{subfig}


% Custom environments
\usepackage[english]{babel}
\usepackage{blindtext}
\usepackage{pifont,mdframed}
\usepackage[dvipsnames]{xcolor}

% graphics
\usepackage{tikz}
\usetikzlibrary{calc,patterns,angles,quotes}

% math
\usepackage{amsmath}
\usepackage{amsthm}

% SI units
\usepackage{siunitx}

% Make margins reasonable
\usepackage{fullpage}
\usepackage{changepage}   % for the adjustwidth environment

% Make TOC clickable
\usepackage{xcolor}
\usepackage{hyperref}

% Graphics
\usepackage{graphicx}
\graphicspath{
	{images/}
}


% References
\usepackage{fancyref}

% Index
\usepackage{mfirstuc}
\usepackage{makeidx}

% Header information
\title{Physics: Notes for Independent Learning}
\author{Jeffrey A. Robinson}

\makeindex
% Add index to TOC
\usepackage[totoc]{idxlayout}

%
%	Defined functions
%

% Theorem, Definitions, Proofts
\theoremstyle{definition}
\newtheorem{definition}{Definition}[chapter]

% Overline and under line text
\makeatletter
\newcommand*{\textoverline}[1]{$\overline{\hbox{#1}}\m@th$}
\makeatother

\makeatletter
\newcommand*{\textunderline}[1]{$\underline{\hbox{#1}}\m@th$}
\makeatother

\makeatletter
\newcommand*{\textbottomtopline}[1]{\textunderline{\textoverline{#1}}}
\makeatother

% Roman Numerals inline
\makeatletter
\newcommand*{\rom}[1]{\expandafter\@slowromancap\romannumeral #1@}
\makeatother

\newcommand{\force}[1]{#1~\si{\newton}}
\newcommand{\velocity}[1]{#1~\si{\meter/\second}}
\newcommand{\acceleration}[1]{#1~\si{\meter/\second^2}}
\newcommand{\mass}[1]{#1~\si{\kilo\gram^2}}
\newcommand{\distance}[1]{#1~\si{\meter}}
\newcommand{\seconds}[1]{#1~\si{\second}}
\newcommand{\hours}[1]{#1~\si{\hour}}

% format and add to index
\newcommand{\addindex}[1]{%
  \textbf{#1}%
  \index{#1@\protect\capitalisewords{#1}}%
   }
\makeatother

\newcommand{\vT}[2]{\left<#1,#2\right>}

% refs
\newcommand{\Def}[1]{Definition~\ref{#1}}
\newcommand{\Fig}[1]{Figure~\ref{#1}}
\newcommand{\Eq}[1]{Equation~\ref{#1}}

%
%	Defined Environments
%
\newenvironment{caution}
  {%
  	\par\begin{mdframed}%
		\begin{list}{}{									%
			\leftmargin=1cm								%
			\labelwidth=\leftmargin						%
		}\item[											%
			\color{BurntOrange}							%
        	\large\textsc{\textbottomtopline{Caution}}	%
        ]												%
  }
  {\end{list}\end{mdframed}\par}


\newcounter{DiscussionCounter}[chapter]
\newcounter{ExerciseCounter}[chapter]

\newcommand{\discussion}[1] {\noindent\stepcounter{DiscussionCounter}\textbf{Q\arabic{chapter}.\arabic{DiscussionCounter}}\vspace{1mm} %
\begin{adjustwidth}{0.5cm}{}#1\end{adjustwidth}%
\vspace{5mm}}

\newcommand{\exercise}[1] {\noindent\stepcounter{ExerciseCounter}\textbf{\arabic{chapter}.\arabic{ExerciseCounter}}\vspace{1mm}%
\begin{adjustwidth}{0.5cm}{}#1\end{adjustwidth}\vspace{5mm} }

\addcontentsline{toc}{section}{Introduction}

\setcounter{secnumdepth}{0}

% Begin
\begin{document}

\frontmatter

% Title and TOC
\maketitle
\tableofcontents

\mainmatter

\part{Mechanics}

\chapter{Units, Physical Quantities, and Vectors}

\section*{Discussion Questions}

% 1.1
\discussion{We just need a single experiment to disprove a theory.  However, there is no number of experiments to prove a theory, just increase confidence.  This breaks down with a finite number of options.}

% 1.2
\discussion{That is not possible as the tangent function takes in degrees and not distances.}

% 1.3
\discussion{My height in imperial is 6' 2", which is 187.96 cm.  My weight is 178 lbs in imperial and therefore it is 792.1 N}

% 1.4
\discussion{Yes.  Currently the mass is derived directly from these.  Other quantities are based off of properties events in nature, but the mass is derived by humans.}
d
% 1.5
\discussion{You could use pulsars as a way to measure time as they have a very precise rotation.  Less precise measures could be position of sun in the sky to determine the hour of day.}

% 1.6
\discussion{Let's assume an 8 by 11 inch sheet of paper.  Then you can cut it into squres of 1x1 inch and then stack them up.  Measure that value and then divide it by 88 to approximate the thickness of a piece of paper.}

% 1.7
\discussion{An axiom is a self consistent statement.  It does not mean that it matches reality.  In terms of physics you could argue that an axiom is only true if the assumptions you are making are true.  A theory is only true as far as the experiments have shown it to be the best match based on observations.}

% 1.8
\discussion{The units of volume are kilogram per meter-cubed and gram per centimeter-cubed.}

% 1.9
\discussion{Joe is neither accurate not precise, Moe is accurate but not precise, Flo is both accurate and precise.}

% 1.10
\discussion{No, the length of $(1,1,1)$ is $\sqrt{3}$.  No, the length of $(3,-2)$ is $\sqrt{10}$}

% 1.11
\discussion{Simple math, $\frac{9.8*60}{0.03^2} \approx{} 653KN/m$, $\frac{9.8*600}{0.092903} \approx{} 63KN/m$, over a 10x difference.}

% 1.12
\discussion{There are no two vectors with different lengths that when summed together are 0.  This is due to the fact of dimensional independence requires that to get 0 the entries in the vector as the same position must have the same magnitude but different sign.  To get three vectors of different lengths you can create three vectors of the form: $\vec{A} = (x,y,0), \vec{B} = (-x,0,z), \vec{C} = (0,-y,-z)$ where $|x| > |y| > |z|$. From here it is easy to show that $|\vec{A}| > |\vec{B}| > |\vec{C}|$.}

% 1.13
\discussion{The speed of light is 299,792,458 m/s and the radius of the earth is 6,400 km, or 6,400,000 m.  Therefore the circumference, 2$\pi{}$r, is 40,212,385.9659, then the circumference divided by the speed of light is 0.1341340814 seconds.}

% 1.14
\discussion{These are not vectors as they have no distance.}

% 1.15
\discussion{Only if dealing with imaginary numbers. Otherwise you will have a magnitude of $\sqrt{A^2 + B^2}$ and the only way that is zero is if A and B are both zero.  As before this would have to involve imaginary numbers.  You can do a proof to show that $\sqrt{\sum A_i^2} \geq |A_i| \forall i$}

% 1.16
\discussion{(a) No. (b) Only in reference to another vector, which points in the opposite direction.  Because all the component values of the two vectors are negatives of each other.  i.e $\vec{A}$ is the negative of $\vec{B}$ if-and-only-if $\forall i~A_i = -B_i$}

% 1.17
\discussion{For the first question $\vec{A}$ and $\vec{B}$ are multiples if each other. For the second question they both have to be the zero vector.}

% 1.18
\discussion{No.  Since neither are the zero vector when using the formulas $|\vec{A}| |\vec{B}| \cos{\alpha}$ and $|\vec{A} \times{} \vec{B}| = |\vec{A}||\vec{B}|\sin{\alpha}$ either $\sin{\alpha}$ or $\cos{\alpha}$ is 0 but not both.}

% 1.19
\discussion{No.  Time has a single component, and you can also do a 1:1 mapping to the reals (trivially) which is defined to be a scalar.}

% 1.20
\discussion{You can work out the math to show it is a unit vector, the direction is the same direction as $\vec{A}$.  Using the identify $\vec{A}/A \cdot \vec{i} = |\vec{A}/A||\vec{i}|cos{\left(\alpha\right)}$ then we have the right side just equal to $cos{\left(\alpha\right)}$.}

% 1.21
\discussion{The percent error is $1 - \frac{890010}{890000} = 1\mathrm{e}{-6}\%$.  You can argue in this case it is correct to display all digits as the exact distances are known.}

% 1.22
\discussion{Assuming 3-dim vectors, (a) Valid (b) Valid (c) Valid (d) Valid (e) No, cannot do cross product between Vector and scalar.}

% 1.23
\discussion{The cross product products a vector which is orthogonal to the original as long as not same vector.  You can just pick $\vec{A}$ and $\vec{B}$ to be the same.  If they are all zero obviously they are same.}

% 1.24
\discussion{Cross product produces vector orthogonal to original two.  The dot product of two orthogonal vectors is 0.}

% 1.25
\discussion{No.  $\vec{A} = (1,0,0)$ and $\vec{B} = (0,1,0)$ will satisfy this.}

% 1.26
% TODO : Figure out what they are asking.
\discussion{I don't know what they are talking about.  Their question makes no sense.}

\section*{Exercises}

% 1.1
\exercise{(a) 2.57 km (b) 1.18 miles}

% 1.2
\exercise{28.9 in$^3$}

% 1.3
\exercise{$5.59\mathrm{e}{-9}$}

% 1.4
\exercise{13.5 g/cm$^3$ $\to$ 13.5$\mathrm{e}{3}$ kg/m$^3$ }

% 1.5
\exercise{5.36 L}

% 1.6
\exercise{6.86 Hectares}

% 1.7
\exercise{34.9 Years older}

% 1.8
% TODO
%\exercise{}

% 1.9
% TODO
%\exercise{}

% 1.10
% TODO
%\exercise{}



\chapter{Motion along a Straight Line}

\section{Discussion Questions}

% 2.1
\discussion{Speed.  Velocity has a direction as well as a magnitude.}

% 2.2
\discussion{Since the dots appear to be increasing by what appears a multiple of the previous distance I would say it is graph (d).}

% 2.3
\discussion{Yes, if the velocity and acceleration have ``different signs''.  This is because the velocity is ``slowing down'' and will eventually reach zero, then start ``increasing again''.  You can also look at the formula for velocity with constant acceleration to see why.  It cannot reverse twice though as a straight line cannot pass through the x-axis twice.}

% 2.4
\discussion{Either when the velocity is constant or special cases when the acceleration changes direction and after a period of time the average velocity will match the instantaneous velocity.}

% 2.5
\discussion{ (b) Yes.  When the velocity and acceleration are in ``opposite directions''.  Yes, if they both point in the ``same direction''.}

% 2.6
\discussion{Constant acceleration.  Anytime you start from 0 speed and accelerate then end up at 0 speed will the speed equal the magnitude of the velocity at some point.}

% 2.7
\discussion{Average velocities are the same but in opposite directions.}

% 2.8
\discussion{You could argue that you would be able to ticket all parked cars that people who speed pass by.}

% 2.9
% This question is no very clear.  I assume they mean instantaneous velocity when they say velocity.
\discussion{No, because displacement is 0, $\Delta{x}$ = zero, then by definition the average velocity is 0.  For the second yes;  $x =sin(t)$, then $y=cos(t)$.  When t = 0 displacement is 0 but velocity is 1.}

% 2.10
\discussion{Yes. If you assume the object is already moving and there is no acceleration then the object will keep moving.}

% 2.11
\discussion{180km/hr * 1h/60m * 1m = 3km. 6.0 seconds is 1/10th of a minute, so 1/10th of 3km, 0.3km}

% 2.12
\discussion{360km/hr, since this is linear half the time means twice the speed.}

% 2.13
\discussion{}

% 2.14
\discussion{}

% 2.15
\discussion{}

% 2.16
\discussion{}

% 2.17
\discussion{}

% 2.18
\discussion{}

% 2.19
\discussion{}

% 2.20
\discussion{}

% 2.21
\discussion{}

% 2.22
\discussion{}

\section{Exercises}

\section{Problems}

\section{Challenge Problems}

\chapter{Motion in Two or Three Dimensions}

\section{Discussion Questions}

% 3.1
\discussion{}

% 3.2
\discussion{}

% 3.3
\discussion{}

% 3.4
\discussion{}

% 3.5
\discussion{}

% 3.6
\discussion{}

% 3.7
\discussion{}

% 3.8
\discussion{}

% 3.9
\discussion{}

% 3.10
\discussion{}

% 3.11
\discussion{}

% 3.12
\discussion{}

% 3.13
\discussion{}

% 3.14
\discussion{}

% 3.15
\discussion{}

% 3.16
\discussion{}

\section{Exercises}

\section{Problems}

\section{Challenge Problems}


\chapter{Newton's Laws of Motion}

\section{Discussion Questions}

\section{Exercises}

\section{Problems}

\section{Challenge Problems}


\chapter{Applying Newton's Laws}

\section{Discussion Questions}

\section{Exercises}


\chapter{Work and Kinetic Energy}

\section{Discussion Questions}

\section{Exercises}

\section{Problems}

\section{Challenge Problems}


\chapter{Potential Energy and Energy Conservation}

\section{Discussion Questions}

\section{Exercises}

\section{Problems}

\section{Challenge Problems}


\chapter{Momentum, Impulse, and Collisions}

\section{Discussion Questions}

\section{Exercises}

\section{Problems}

\section{Challenge Problems}


\chapter{Rotation of Rigid Bodies}

\section{Discussion Questions}

\section{Exercises}

\section{Problems}

\section{Challenge Problems}


\chapter{Dynamics of Rotational Motion}

\section{Discussion Questions}

\section{Exercises}


\chapter{Equilibrium and Elasticity}

\section{Discussion Questions}

\section{Exercises}


\chapter{Fluid Mechanics}

\section{Discussion Questions}

\section{Exercises}


\chapter{Gravitation}

\section{Discussion Questions}

\section{Exercises}


\chapter{Periodic Motion}

\section{Discussion Questions}

\section{Exercises}

\section{Problems}

\section{Challenge Problems}


% Waves need to be understood 100% Almost all things in Physics are modeled using waves.
\part{Waves and Acoustics}

\chapter{Mechanical Waves}

\section{Discussion Questions}

\section{Exercises}


\chapter{Sound and Hearing}

\section{Discussion Questions}

\section{Exercises}


% Fundamental
\part{Thermodynamics}

\chapter{Temperature and Heat}

\section{Discussion Questions}

\section{Exercises}

\section{Problems}

\section{Challenge Problems}


\chapter{Thermal Properties of Matter}

\section{Discussion Questions}

\section{Exercises}

\section{Problems}

\section{Challenge Problems}


\chapter{The First Law of Thermodynamics}

\section{Discussion Questions}

\section{Exercises}


\chapter{The Second Law of Thermodynamics}

\section{Discussion Questions}

\section{Exercises}

\section{Problems}

\section{Challenge Problems}


% Advanced physics comes out of this!
\part{Electromagnetism}

\chapter{Electric Charge and Electric Field}

\section{Discussion Questions}

\section{Exercises}


\chapter{Gauss's Law}

\section{Discussion Questions}

\section{Exercises}

\section{Problems}

\section{Challenge Problems}


\chapter{Eletric Potential}

\section{Discussion Questions}

\section{Exercises}


\chapter{Capacitance and Dialetrics}

\section{Discussion Questions}

\section{Exercises}

\section{Problems}

\section{Challenge Problems}


\chapter{Current, Resistance, and Electromotive Force}

\section{Discussion Questions}

\section{Exercises}


\chapter{Direct-Current Circuits}

\section{Discussion Questions}

\section{Exercises}

\section{Problems}

\section{Challenge Problems}


\chapter{Magnetic Field and Magnetic Forces}

\section{Discussion Questions}

\section{Exercises}

\section{Problems}

\section{Challenge Problems}


\chapter{Sources of Magnetic Field}

\section{Discussion Questions}

\section{Exercises}


\chapter{Electromagnetic Inducution}

\section{Discussion Questions}

\section{Exercises}

\section{Problems}

\section{Challenge Problems}


\chapter{Inductance}

\section{Discussion Questions}

\section{Exercises}

\section{Problems}

\section{Challenge Problems}


\chapter{Alternating Currents}

\section{Discussion Questions}

\section{Exercises}


\chapter{Electromagetic Waves}

\section{Discussion Questions}

\section{Exercises}

\section{Problems}

\section{Challenge Problems}


% More applied like stuff
\part{Optics}

\chapter{The Nature and Propagation of Light}

\section{Discussion Questions}

\section{Exercises}


\chapter{Geometric Optics}

\section{Discussion Questions}

\section{Exercises}


\chapter{Interference}

\section{Discussion Questions}

\section{Exercises}


\chapter{Diffraction}

\section{Discussion Questions}

\section{Exercises}

\section{Problems}

\section{Challenge Problems}


% Introduction to Relativity, particle physics, quantum physics
\part{Modern Physics}

\chapter{Relativity}

\section{Discussion Questions}

\section{Exercises}

\section{Problems}

\section{Challenge Problems}


\chapter{Photons: Light Waves Behaving as Particles}

\section{Discussion Questions}

\section{Exercises}


\chapter{Particles Behaving as Waves}

\section{Discussion Questions}

\section{Exercises}

\section{Problems}

\section{Challenge Problems}


\chapter{Quantum Mechanics I: Wave Functions}

\section{Discussion Questions}

\section{Exercises}


\chapter{Quantum Mechanics II: Atomic Structure}

\section{Discussion Questions}

\section{Exercises}


\chapter{Molecules and Condensed Matter}

\section{Discussion Questions}

\section{Exercises}

\section{Problems}

\section{Challenge Problems}


\chapter{Nuclear Physics}

\section{Discussion Questions}

\section{Exercises}

\section{Problems}

\section{Challenge Problems}


\chapter{Particle Physics and Cosmology}

\section{Discussion Questions}

\section{Exercises}

\section{Problems}

\section{Challenge Problems}


\backmatter

% Appendix stuff
\appendix 
\part{Appendices}

\label{app:Constants}
\chapter{Constants}

\begin{tabular}{l l}
	Speed of Light	& : 299,792,458				\\
	Mass Earth 		& : 5.972E24 kg	\\
	Mass Sun		& : 1.989E30 kg
\end{tabular}


\label{app:SI}
\chapter{SI Units and Prefixes}

\section{Units}

\begin{tabular}{l l r}
Name 	& Unit Name & Unit 		\\ \hline
Time 	& Second 	& s			\\
Length 	& Meter 	& m			\\
Mass	& Kilogram	& kg		\\
Energy	& Joule		& $\frac{kg\cdot{}m}{s}$
\end{tabular}

\section{Prefixes}

\begin{tabular}{l | l}
	Prefix 	& Multiplier	\\ \hline
	Tera- 	& 1E12 			\\
	Giga- 	& 1E9 			\\
	Mega- 	& 1E6 			\\
	Kilo- 	& 1E3 			\\
	hecto-	& 1E2			\\
	Deca- 	& 1E1 			\\
	Deci-	& 1E-1 			\\
	Centi- 	& 1E-2 			\\
	Milli- 	& 1E-3 			\\
	Micro-	& 1E-6 			\\
	Nano- 	& 1E-9 			\\
	Pico- 	& 1E-12 		\\
	Femto- 	& 1E-15 		\\
	Atto- 	& 1E-18 		\\
	Zepto- 	& 1E-21 		\\
	Yocto- 	& 1E-24
\end{tabular}

\printindex

\end{document}