
\chapter{Applying Newton's Laws}

\section{Discussion Questions}

% 5.1
\discussion{
	The tension (T) is the combined weight of the man and seat.  On each end of the rope there is a force equal to half the tension.
}

% 5.2
\discussion{
	If the only forces are the normal force and gravity then they are only equal if the ground you stand on is perfectly flat.  Two times they are not equal is when you are standing on a hill and when an elevator first starts moving.  When they are equal is when you are standing in an elevator not moving and when you are sitting down.
}

% 5.3
\discussion{
	The force of gravity pulls the rope down.  If there was no sag then that means a force is counter acting gravity.  If the string is straight then the poles have no force in the y direction to counter balance the force of gravity.  Since the only forces come from gravity and the poles there is no way that the rope can be perfectly flat.
}

% 5.4
\discussion{
	The external forces are: gravity, normal force, friction between tires and ground, and wind resistance. 
}

% 5.5
\discussion{
	(1) Have a device which pushes you at a set acceleration into a scale. \\
	(2) Attach pole to a spring, then have astronaut spin at end of pole for a set radial velocity that causes acceleration to be the same as earths. \\
	(3) Similar to \#2, spin a room so that it mimics gravity, then stand on scale.
}

% 5.6
\discussion{
	Pushing horizontal.  Because some of the force is being transfered into the ramp, which is then pushing back at you.
}

% 5.7
\discussion{
	It is falling at $\acceleration{9.8}$.  She is probably going to die.
}

% 5.8
\discussion{
	Down the plane, because gravity is helping in that direction.
}

% 5.9
\discussion{
	A terminal speed if no net forces.  When they open their parachute there will be a large force opposing gravity, then tapers off as it approaches new (much smaller) terminal speed.
}

% 5.10
\discussion{
	Friction.  Pushing down increase normal force which in turn increases the force of friction.  Pushing up reduces the normal force which in turn decreases friction.
}

% 5.11
\discussion{
	Air resistance.
}

% 5.12
\discussion{
	Smaller.  Air resistance is dependent on the area which is perpendicular to the velocity.
}

% 5.13
\discussion{
	Lost force - Elevator accelerating upwards (fights gravity).  This causes the normal force to increase.
	Least force - Elevator accelerating downward (works with gravity).  This causes the normal force to decrease.
}

% 5.14
\discussion{
	{\bf Note:} I think this question is poorly worded. \\
	(a) Static friction is what makes the car move. \\
	(b) I am not sure of anything that Kinetic friction {\it causes} motion.
}

% 5.15
\discussion{
	To increase friction and gas efficiency.
}

% 5.16
\discussion{
	Spin out.  The coefficient of both static and kinetic friction has decreased.
}

% 5.17
\discussion{
	I think it will lag behind because as the ball moves upwards gravity will be slowing it down, but your hand will keep rotating at the same speed.  So below the horizon.
}

% 5.18
\discussion{
	This is not a real force.
}

% 5.19
\discussion{
	No.  The velocity is already perpendicular to her face so it will fly straight up.
}

% 5.20
\discussion{
	As the roller coaster moves up gravity will be counter acting gravity, so at the bottom the radius will need to be larger so that the force of gravity plus the new radial forces are within safety limits.
}

% 5.21
\discussion{
	If we assume photograph is same time delta from start of fall and far enough along to determine any differences then yes.  The one with air will have moved a smaller distance due to air resistance.
}

% 5.22
\discussion{
	The one with deeper treads.  It will have larger friction.
}

% 5.23
\discussion{
	Should be same, just opposite directions.
}

% 5.24
\discussion{
	(i) Same time (ii) heavier hits first as it has more ability to push air out the way.
}

% 5.25
\discussion{
	(d) Since the acceleration will approach 0 as we reach terminal velocity.
}

% 5.26
\discussion{
	(a) We will start off slow then eventual reach terminal velocity.  Air resistance is also non-linear.
}

% 5.27
\discussion{
	From hitting to maximum.  This is due to the fact that $V_x$ is decreasing.  I assume that we don't hit terminal velocity in terms of $V_y$ so we should expect the same change in distance going up and down.
}

% 5.28
\discussion{
	Assuming enough distance (technically infinite) then yes.  This is because gravity is always at work until it hits the ground but there is no horizontal force except for air resistance which is slowly reducing $V_x$ to zero.
}


\section{Exercises}

% 5.1
\exercise{
	
}

% 5.2
\exercise{
	
}

% 5.3
\exercise{
	
}

% 5.4
\exercise{
	
}

% 5.5
\exercise{
	
}

% 5.6
\exercise{
	
}

% 5.7
\exercise{
	
}

% 5.8
\exercise{
	
}

% 5.9
\exercise{
	
}

% 5.10
\exercise{
	
}

% 5.11
\exercise{
	
}

% 5.12
\exercise{
	
}

% 5.13
\exercise{
	
}

% 5.14
\exercise{
	
}

% 5.15
\exercise{
	
}

% 5.16
\exercise{
	
}

% 5.17
\exercise{
	
}

% 5.18
\exercise{
	
}

% 5.19
\exercise{
	
}

% 5.20
\exercise{
	
}

% 5.21
\exercise{
	
}

% 5.22
\exercise{
	
}

% 5.23
\exercise{
	
}

% 5.24
\exercise{
	
}

% 5.25
\exercise{
	
}

% 5.26
\exercise{
	
}

% 5.27
\exercise{
	
}

% 5.28
\exercise{
	
}

% 5.29
\exercise{
	
}

% 5.30
\exercise{
	
}

% 5.31
\exercise{
	
}

% 5.32
\exercise{
	
}

% 5.33
\exercise{
	
}

% 5.34
\exercise{
	
}

% 5.35
\exercise{
	
}

% 5.36
\exercise{
	
}

% 5.37
\exercise{
	
}

% 5.38
\exercise{
	
}

% 5.39
\exercise{
	
}

% 5.40
\exercise{
	
}

% 5.41
\exercise{
	
}

% 5.42
\exercise{
	
}

% 5.43
\exercise{
	
}

% 5.44
\exercise{
	
}

% 5.45
\exercise{
	
}

% 5.46
\exercise{
	
}

% 5.47
\exercise{
	
}

% 5.48
\exercise{
	
}

% 5.49
\exercise{
	
}

% 5.50
\exercise{
	
}

% 5.51
\exercise{
	
}

% 5.52
\exercise{
	
}

% 5.53
\exercise{
	
}

% 5.54
\exercise{
	
}

% 5.55
\exercise{
	
}

% 5.56
\exercise{
	
}

% 5.57
\exercise{
	
}

% 5.58
\exercise{
	
}

% 5.59
\exercise{
	
}


\section{Problems}

\section{Challenge Problems}
