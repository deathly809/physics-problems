
\chapter{Newton's Laws of Motion}

\section{Discussion Questions}

% 4.1
\discussion{A non-inertial frame is one in which Newton's first law does not hold.  Example given in book is roller-skater on a bus.  The second and third law should hold.}

% 4.2
\discussion{no air-resistance.}

% 4.3
\discussion{Because they don't have any force acting on those parts of the body so they don't start moving until they tense up or the seat hits them.}

% 4.4
\discussion{no acceleration so no force.}

% 4.5
\discussion{Because gravity is also at play.}

% 4.6
\discussion{It goes in a straight line target to circle at point let go. Eventually falls to the ground due to gravity.}

% 4.7
\discussion{Same as Q4.3,  They want to stay at the current velocity but car is acceleration.}

% 4.8
\discussion{Those are not forces.  Really describing velocity and Newton's first law.}

% 4.9
\discussion{Bus is accelerating or is on a hill.  There is no way to tell which.  }

% 4.10
\discussion{$N * s^2/m$}

% 4.11
\discussion{Earth is rotating on axis which is also rotating around sun which is rotating inside galaxy.}

% 4.12
\discussion{Yes.  The person will feel a force.}

% 4.13
\discussion{No, this is the result of the force.}

% 4.14
\discussion{$\acceleration{9.8}$}

% 4.15
\discussion{No, the ball will want to keep going in the same direction (relative to earth) and will appear to curve to the players.}

% 4.16
\discussion{Force is in newtons and the units don't match.  Also, the force of gravity is not technically constant.}

% 4.17
\discussion{
	(i) By Newton's second and third law.  The mass of the rock is much larger than that of you foot.  Therefore the acceleration your foot feels will be much higher than the rock.  \\
	(ii) No, you can remove the pain if you were to wear a shoe then the acceleration would be less, and also there would be cushion.
}

% 4.18
\discussion{Imagine jumping off a couch.  You crouch when you land with an acceleration away from the ground which is a small applied force over time.  Now imaging when you jump of a roof, you would need to accelerate much faster, so a larger force.}

% 4.19
\discussion{When jumping into water the water ``gives'' as you hit.  In essence you are applying a much smaller force over a longer period of time.  When you hit the earth you can't spread out the force on your body so you break your legs.}

% 4.20
\discussion{The affects of gravity become less influential.}

% 4.21
\discussion{When you double the amount of time the acceleration is $\frac{1}{4}^{th}$ the value.  This is not enough force to break the bonds on the string.}

% 4.22
\discussion{No, only if the same direction.}

% 4.23
\discussion{The stone with more mass.  Imagine you split the heavy stone in half, so you have two stones which each are the same mass as the first.  You then hold all three up at the same level and let go.  All three will hit the ground at the same time.  What is the difference between the stones being connected vs. being unconnected?  Nothing.  Also, you can think of gravity pulling on more stone, but at the same ``acceleration''.}

% 4.24
\discussion{kg is a mass, lb is a force.}

% 4.25
\discussion{The friction between the horse and the ground is much higher than the friction between the wagon and the ground.  Because of that the horse is moving forward and taking the wagon with it.}

% 4.26
\discussion{False.  Newton's third law.}

% 4.27
\discussion{The forces will be the same.  The accelerations on the other hand will not be.}

% 4.28
\discussion{
	(i) Friction from the road, engine, and other car components along with air resistance slows a car down. \\
	(ii)  Chemcial reactions are causing small explosions which are creating a force in the engine which is, inefficiently, transmitted to the wheel which then use the frictional force to accelerate the car.
}

% 4.29
\discussion{The force between the van and car are the same due to Newton's third law.  The total forces on the van would be larger than that on the car.  If we look at it as a closed system and the acceleration is the same on both vehicles then the force on the car has to be smaller as the mass is smaller than that of the van.}

% 4.30
\discussion{The force on the rope is the same but the friction on the ground for the two groups is not.  The weight of the groups matters as well as the strength of the occupants (hold rope, push against ground).}

% 4.31
\discussion{Less, $\force{100}$ be on the entire system. If we only look at each box then we see that the box with more ``weight'' has a higher force acting on it.}

% 4.32
\discussion{Does not sound like it.  Since the velocity appears to be constant the acceleration would be 0.}

% 4.33
\discussion{The water travels down your finger and flies off.  Also, the force from your hands  }

% 4.34
\discussion{the blood requires a force to move it up, but because blood is a liquid it can flow easier than a solid.  The blood is pushed by the veins and arteries and the blood pushes back as well.  The veins and arteries expand because of this push back so not all the blood moves up with you.}

% 4.35
\discussion{When the car is hit from behind you are pushed hard forward (Force) and then when you hit the seatbelt it pulls hard on your body (Force) but your head keeps going since it is on your neck but eventually your head will be pulled back by your neck and then you will start moving back into the seat.}

% 4.36
\discussion{With no seatbelt you will have no force acting on your body as the car stops.  Then you will hit the windshield (or steering wheel) and then have a force acting on your head.  If you have enough velocity you can make it through the windshield and then you will be stopped by either the car in front of you are the ground.}

% 4.37
\discussion{Same force because of Newton's third law.  Compact car has greatest acceleration due to Newton's Second law.  Therefore the occupants will have a larger acceleration acting on them compared to the larger vehicle, and therefore larger force.}

% 4.38
\discussion{(
	(a) Can't \\
	(b) You would move towards the end of a rocket.
}

\section{Exercises}

\subsection*{Force and Interactions}

% 4.1
\exercise{
	553N @ $\deg{27.2}$	
}

% 4.2
\exercise{
	866N @ $\deg{78.1}$	
}

% 4.3
\exercise{
	T = $\force{3.15}$
}

% 4.4
\exercise{
	
}

% 4.5
\exercise{
	
}

% 4.6
\exercise{
	
}

% 4.7
\exercise{
	
}

% 4.8
\exercise{
	
}

% 4.9
\exercise{
	
}

% 4.10
\exercise{
	
}

% 4.11
\exercise{
	
}

% 4.12
\exercise{
	
}

% 4.13
\exercise{
	
}

% 4.14
\exercise{
	
}

% 4.15
\exercise{
	
}

% 4.16
\exercise{
	
}

% 4.17
\exercise{
	
}

% 4.18
\exercise{
	
}

% 4.19
\exercise{
	
}

% 4.20
\exercise{
	
}

% 4.21
\exercise{
	
}

% 4.22
\exercise{
	
}

% 4.23
\exercise{
	
}

% 4.24
\exercise{
	
}

% 4.25
\exercise{
	
}

% 4.26
\exercise{
	
}

% 4.27
\exercise{
	
}

% 4.28
\exercise{
	
}

% 4.29
\exercise{
	
}

% 4.30
\exercise{
	
}

% 4.31
\exercise{
	
}

% 4.32
\exercise{
	
}

% 4.33
\exercise{
	
}

% 4.34
\exercise{
	
}

% 4.35
\exercise{
	
}

% 4.36
\exercise{
	
}

% 4.37
\exercise{
	
}

% 4.38
\exercise{
	
}

\section{Problems}

\section{Challenge Problems}
