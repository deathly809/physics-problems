
\chapter{Work and Kinetic Energy}
This work is in notebooks.

\section{Discussion Questions}

% 6.1
\discussion{
    No. Work depends on the direction of the Force and Distance. Flipping the axie will flip the signs for both.
}

% 6.2
\discussion{
    There is no work being done. We are converting some "store" of energy into potential energy.
}

% 6.3
\discussion{
    Yes. We are simplifying this by ignoring energy in air, friction, etc.
}

% 6.4
\discussion{
    Since we are assuming we start with $v_0=0$ then w = $\frac{mv^2}{2}$ after reaching a velocity (speed) of v.
    2w = $\frac{m(v')^2}{2}$, $\sqrt{2}v$, 2k
}

% 6.5
\discussion{
    Circular motion seems like a candidate. An example would be a ball on a rope going in a circle around a central point.
}

% 6.6
\discussion{
    Should be the same magnitude but positve for cart-rope, and negative for rope-bucket.
}

% 6.7
\discussion{
    There is the force of gravity and on from the rope on the pendulum. However, there is no movement along those lines. This means that there is no work being done.
}

% 6.8
\discussion{
    They all have the same speed and also the same amount of work. This comes fromt the equation $$\frac{mv^2}{2} = mgh$$ and the definition of Work.
}

% 6.9
\discussion{
    You just need the starting and ending velocity to be the same. This is only possible for a non-continuous function of F as you would have to have some postiive force then instantly switch to a negative force.
}

% 6.10
\discussion{
    The system cannot lose energy, but a specific object can. For instance if you are increasing the speed of a block,
    and then decrease the speed back to the starting value then you do zero work, but obviously energy was used. The
    same if you move something away from a position and then back to it again.
}

% 6.11
\discussion{

}

% 6.12
\discussion{

}

% 6.13
\discussion{

}

% 6.14
\discussion{

}

% 6.15
\discussion{

}

% 6.16
\discussion{

}

% 6.17
\discussion{

}

% 6.18
\discussion{

}

% 6.19
\discussion{

}

% 6.20
\discussion{

}

% 6.21
\discussion{

}

% 6.22
\discussion{

}

% 6.23
\discussion{

}

% 6.24
\discussion{

}

\section{Exercises}

\section{Problems}

\section{Challenge Problems}
