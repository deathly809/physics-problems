
\chapter{Electric Charge and Electric Field}

\section{Formulas and Values}

mass of electron = $m_e$ = $9.10938291(45)\cdot 10^{-31}kg$ \\
\noindent
mass of proton = $m_e$ = $1.672621777(74)\cdot 10^{-27}kg$ \\
\noindent
mass of neutron = $m_e$ = $1.674927351(74)\cdot 10^{-27}kg$ \\
\noindent
k = $\frac{1}{4\pi\epsilon}$ = $10^{-7} \cdot c^2$ = $8.987551787\cdot10^9 N \cdot m^2 / C^2$ \\
\noindent
charge = e = $1.602176565 \cdot 10^{-19}$C\\
\noindent
permittivity of free space = $\epsilon_{0} = 8.854\cdot10^{-12}C^2/N \cdot m^2$ \\
\noindent
$\bf{Coulomb's law}$ \\
F = $\frac{|q_1q_2|}{4\pi\epsilon_0r^2}$

\section{Discussion Questions}

% 21.1
\discussion{
	If they come from the same tape then they should have an electrical uniform charge density.  If you stick them together, sticky to non-sticky, and pull off I assume what happens is that some of the charge is left on the non-sticky side due to the loss of glue.
}

% 21.2
\discussion{
	There are two possible configurations. let the spheres be labeled A and B, then either A has positive charge and B has a negative charged, or vice-versa.  They could cling if the charge was actually located inside the ball and is insulated from the exterior shell.
}

% 21.3
\discussion{
	$\bf{TODO}$ This entirely depends on the constant force between charge particles.  Too large or small and atoms could not exist as either they explode or electrons would ...
}

% 21.4
\discussion{
	There are a few facts: (1) Charge seems to want to stay on clothes instead of metal (2) We start out at an equal distribution of charge (3) Some material has a positive charge, others negative.  From this it would seem that electrons are leaving some of the material and entering other materials.  This would imply that some materials will accept more charge easier than others.  
}

% 21.5
\discussion{
	In the beginning the sphere has no charge, but when you bring a charged rod (doesn't matter if positive or negative charge) the sphere will be pulled close due to induction.  When they touch the charge will even out, and in this case that means both will become positively charged.  This will cause the repulsion.
}

% 21.6
\discussion{
	I weight about 165 lbs, so about 75 kg.  Let's assume no ions, so each atom will have the same number of neutrons, electrons, and protons.  Then if we sum up the mass of these three and divide by mass by that you can get $$\frac{65kg}{3.34846\cdot10^{-27kg}} = 1.94119\cdot10^{28}$$ this means that I have $6.47063*10^{27}$ of each particle. Since there are approximately an equal number of electrons as protons then I am pretty much neutral.
}

% 21.7
\discussion{
	(a) The top and bottom are positive, the middle negative. Because positive charges have outward electric fields and negative charges have inward electric fields. (b) At infinity since any other point will have one charge pulling and the other charge pushing.
}

% 21.8
\discussion{
	Heat is just the kinetic energy of atoms and particles.  If you can easily move electrons then it is ``easy'' to increase the heat of electrons.
}

% 21.9
\discussion{
	(a) It will follow the electric field line.  Because the vector force points away (or towards) the charge on that line. (b) Yes.  Because the force is along the field lines.
}

% 21.10
\discussion{
	
}

% 21.11
\discussion{
}

% 21.12
\discussion{
}

% 21.13
\discussion{
}

% 21.14
\discussion{
}

% 21.15
\discussion{
}

% 21.16
\discussion{
}

% 21.17
\discussion{
}

% 21.18
\discussion{
}

% 21.19
\discussion{
}

% 21.20
\discussion{
}

% 21.21
\discussion{
}

% 21.22
\discussion{
}

% 21.23
\discussion{
}

\section{Exercises}

\section{Problems}

\section{Challenge Problems}